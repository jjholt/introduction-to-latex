\subsection{SI Units}
The package \href{https://mirror.ox.ac.uk/sites/ctan.org/macros/latex/contrib/siunitx/siunitx.pdf}{\texttt{siunitx}} was briefly mentioned when we discussed circuits.
Generally this is the go-to package for proper typesetting of units both in text and math modes.

\begin{figure}[h]
\centering
\begin{minipage}{0.45\textwidth}
    \si{\micro\gram\per\second} and \SI{1.25}{\kilo\joule}

    Some are made easier: \SIrange{5}{10}{\kg}, \si{\ug}
\end{minipage}
\hfill
\begin{minipage}{0.45\textwidth}
\begin{lstlisting}
\si{\micro\gram\per\second} and \SI{1.25}{\kilo\joule}

Some are made easier: \SIrange{5}{10}{\kg}, \si{\ug}
\end{lstlisting}
\end{minipage}

\end{figure}
\subsection{Chemistry}
\subsubsection{Formulae}
The package \href{https://texdoc.org/serve/mhchem/0}{\texttt{mhchem}} is the go-to for chemical formulae with very simple syntax:
\begin{figure}[h]
\centering
\begin{minipage}{0.49\textwidth}\centering
    \ce{H2CO3 <=> 2H+(aq) + CO3^{2-} + (1/2) H^{.}}

    \ce{H_2 + ^{227}_{90}Th+ ->[\Delta] NO_x}

    A triple bond is \ce{H-C#C-H}
\end{minipage}
\hfill
\begin{minipage}{0.49\textwidth}
    \begin{lstlisting}
\usepackage{mhchem}
\begin{document}
    \ce{H2CO7 <-> 2H+(aq) + CO3^{2-} + H^{.}}

    \ce{H_2 + ^{227}_{90}Th+ ->[\Delta] NO_x}

    A triple bond is \ce{H-C#C-H}
\end{document}
\end{lstlisting}
\end{minipage}
\end{figure}

\subsubsection{Geometry}
For drawing geometry, the package used is \href{https://anorien.csc.warwick.ac.uk/mirrors/CTAN/macros/generic/chemfig/chemfig-en.pdf}{\texttt{chemfig}}.
It is extremely powerful, and the documentation is worth exploring if you need to typeset geometry.
A few examples will be shown below.

\begin{figure}[h]
\centering
\begin{minipage}{0.45\textwidth}
    \chemfig{A>:B-[2]C}
    \qquad \qquad
    \chemfig{CH_3CH_2-[:-60,,3]C(-[:-120]H_3C)=C(-[:-60]H)-[:60]C|{(CH_3)_3}}
\end{minipage}
\hfill
\begin{minipage}{0.45\textwidth}
\begin{lstlisting}
\usepackage{chemfig}
\begin{document}
    \chemfig{A>:B-[2]C}
    \qquad \qquad
    \chemfig{CH_3CH_2-[:-60,,3]C(-[:-120]H_3C)=C(-[:-60]H)-[:60]C|{(CH_3)_3}}
\end{document}
\end{lstlisting}
\end{minipage}
\end{figure}