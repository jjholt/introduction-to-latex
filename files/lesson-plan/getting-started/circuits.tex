We draw circuits using the \texttt{circuitikz} package.
All of the knowledge from \texttt{Tikz} will come in handy, with the benefit that electrical components have already been created and given names.
The \href{https://mirror.ox.ac.uk/sites/ctan.org/graphics/pgf/contrib/circuitikz/doc/circuitikzmanual.pdf}{documentation} has a component list, and you can find the name for the exact component you need.

Let's create our first circuit:
\begin{figure}[h]
\centering
\begin{minipage}{0.39\textwidth}\centering
    \begin{circuitikz}
        \draw (0,0) to[sV] (0,3) -- (3,3) to[R] (3,0) -- (0,0);
    \end{circuitikz}
\end{minipage}
\hfill
\begin{minipage}{0.59\textwidth}
\begin{lstlisting}
\usepackage{circutikz}
\begin{document}
    \begin{circuitikz}
        \draw (0,0) to[sV] (0,3) -- (3,3) to[R] (3,0) -- (0,0);
    \end{circuitikz}
\end{document}
\end{lstlisting}    
\end{minipage}
\end{figure}

To add values with units, the suggestion is to use the package \texttt{siunitx}.
In fact, this is recommended across any document.
Basic usage is \verb|\SI{value}{unit}|, and we have the option to either describe the units in text or simply using a symbol: \verb|\ohm| or \verb|ml|.

You have a couple of options with how to produce the label, \texttt{[element-name=label]}, or \texttt{[element-name, a=label]}.
It automatically chooses a good position, but it can be tweaked with \verb|a=| for one side, and \verb|l=| on the other.

\begin{figure}[h]
    \centering
    \begin{minipage}{0.39\textwidth}\centering
        \begin{circuitikz}
            \draw (0,0) to[sV=\SI{5}{\volt}] (0,3) -- (3,3) to[R, a=\SI{100}{\ohm}] (3,0) -- (0,0);
        \end{circuitikz}
    \end{minipage}
    \hfill
    \begin{minipage}{0.59\textwidth}
    \begin{lstlisting}
\usepackage{circutikz}
\usepackage{siunitx}
\begin{document}
    \begin{circuitikz}
        \draw (0,0) to[sV=\SI{5}{\volt}] (0,3) -- (3,3) to[R=\SI{100}{\ohm}] (3,0) -- (0,0);
    \end{circuitikz}
\end{document}
    \end{lstlisting}    
    \end{minipage}
\end{figure}

Next let's add an amplifier.
By default the inverting input is on the top, hence the option \texttt{noinv input up}.
Remember, \verb|++ (x,y)| creates relative coordinates, so \verb|(1,1) -- ++ (2,0)| means a line from (1,1) to (3,1)

\begin{figure}[h]
\centering
    \begin{circuitikz}
        \draw (0,0) to[sV=\SI{5}{\volt}] (0,3) -- (3,3) to[R, a=\SI{100}{\ohm}] (3,0) -- (0,0);
        \draw (3,3) to[short,*-] ++ (2,0) node [op amp, noinv input up, anchor=+](opamp){};
        \draw (3,0) to[R,-*] (opamp.- |- 3,0) coordinate (intersection) to[R] (opamp.out |- intersection) -- (opamp.out) to[short,-o] ++ (0.2,0) node[above]{ \( v_{out} \) };
        \draw (opamp.-) -- (intersection);
    \end{circuitikz}
\end{figure}

\begin{lstlisting}
\begin{circuitikz}
    \draw (0,0) to[sV=\SI{5}{V}] (0,3) -- (3,3) to[R, a=\SI{100}{\ohm}] (3,0) -- (0,0);
    \draw (3,3) to[short,*-] ++ (2,0) node [op amp, noinv input up, anchor=+](opamp){};
    \draw (3,0) to[R,-*] (opamp.- |- 3,0) coordinate (intersection) to[R] (opamp.out |- intersection) -- (opamp.out) to[short,-o] ++ (0.2,0) node[above]{ \( v_{out} \) };
    \draw (opamp.-) -- (intersection);
\end{circuitikz}
\end{lstlisting}    

The usage of \texttt{|-} and \texttt{-|} may take some getting used to.
\verb!(opamp.out |- intersection)! means ``the coordinates for the intersection between \texttt{opamp.out} and \texttt{intersection}''.
However, \texttt{(0,0) |- (1,1)} creates the lines \texttt{(0,0) -- (0,1) -- (1,1)}; while \texttt{(0,0) |- (1,1)} creates the line \texttt{(0,0) -- (1,0) -- (1,1)}.

In the same way that you can create nodes at any point to include text or an image (in this case, the amplifier), you can also give the coordinates a name.
Bear in you mind you can also create the amplifier with \verb|\node|, as per the example below.

It may have been clear from previous examples, but you don't need to create separate \verb|\draw|.
The motivation to do so is tidiness and legibility.

\begin{figure}[h]
    \begin{minipage}{0.45\textwidth}
        \begin{circuitikz}
            \node [op amp] (opamp) at (0,0) {};
            \draw
                (opamp.out) |- ++ (-1,1.5) to[R] ++ (-2,0) |- (opamp.-) to [R] ++ (-3,0) node[ground]{}
                (opamp.+) to [short,-o] ++ (-1,0) node[left]{$V_{in}$}
            ;
            \draw (opamp.out) to[short,-o] ++(0.5,0) node[right]{$V_{out}$};
        \end{circuitikz}
    \end{minipage}
    \hfill
    \begin{minipage}{0.45\textwidth}
\begin{lstlisting}
\begin{circuitikz}
\node [op amp] (opamp) at (0,0) {};
\draw
    (opamp.out) |- ++ (-1,1.5) to[R] ++ (-2,0) |- (opamp.-) to [R] ++ (-3,0) node[ground]{}
    (opamp.+) to [short,-o] ++ (-1,0) node[left]{$V_{in}$}
;
\draw (opamp.out) to[short,-o] ++(0.5,0) node[right]{$V_{out}$};
\end{circuitikz} 
\end{lstlisting}
    \end{minipage}
\end{figure}

Below you can see an example with only absolute coordinates.

\begin{figure}[h] \centering
    \begin{minipage}{0.44\textwidth}
    \begin{circuitikz}[scale=0.8]
        \draw (0,0.5) to[short, o-*] (0.65,0.5) |- (1.3,1) to[R=R] (3.3,1) to[R=R] (6.3,1) -- (6.3,0.5) to[short, *-o] (7,0.5);
        \draw (0.65,0.5) -- (0.65,0) to[C,a=C] (3.3,0) to[C,a=C] (6.3,0) -- (6.3,0.5);
        \draw (0,-3.5) to[short,o-o] (7,-3.5); %ground
        \draw (2.7,0) to[resistor, a=$\frac{R}{2}$, *-*] (2.7,-3.5);
        \draw (3.8,-3.5) node[circ]{} to[C, a=2C] (3.8,-0.2) -- ++ (0,0) arc (-90:90:0.2) -- (3.8,1) node[circ]{};
        \draw [|->] (0,-3.35) -- (0,0.35) node[midway,fill=white]{$V_{in}$};
        \draw [|->] (7,-3.35) -- (7,0.35) node[midway,fill=white]{$V_{out}$};
    \end{circuitikz}
\end{minipage}
\hfill
\begin{minipage}{0.55\textwidth}
\begin{lstlisting}
\begin{circuitikz}
    \draw (0,0.5) to[short, o-*] (0.65,0.5) |- (1.3,1) to[R=R] (3.3,1) to[R=R] (6.3,1) -- (6.3,0.5) to[short, *-o] (7,0.5);
    \draw (0.65,0.5) -- (0.65,0) to[C,a=C] (3.3,0) to[C,a=C] (6.3,0) -- (6.3,0.5);
    \draw (0,-3.5) to[short,o-o] (7,-3.5); %ground
    \draw (2.7,0) to[resistor, a=$\frac{R}{2}$, *-*] (2.7,-3.5);
    \draw (3.8,-3.5) node[circ]{} to[C, a=2C] (3.8,-0.2) -- ++ (0,0) arc (-90:90:0.2) -- (3.8,1) node[circ]{};
    \draw [|->] (0,-3.35) -- (0,0.35) node[midway,fill=white]{$V_{in}$};
    \draw [|->] (7,-3.35) -- (7,0.35) node[midway,fill=white]{$V_{out}$};
\end{circuitikz}
\end{lstlisting}
\end{minipage}
\end{figure}

It's quite easy to get lost with complicated circuits, so the suggestion is to think about it in terms of a continuous line drawing.
You will notice this circuit drawns the horizontal lines across the the resistors, then across the capacitors, then the vertical lines. 