We draw circuits using the \texttt{circuitikz} package.
All of the knowledge from \texttt{Tikz} will come in handy, with the benefit that electrical components have already been created and given names.
The \href{https://mirror.ox.ac.uk/sites/ctan.org/graphics/pgf/contrib/circuitikz/doc/circuitikzmanual.pdf}{documentation} has a component list, and you can find the name for the exact component you need.

\subsubsection{Our first circuit}
% Let's create our first circuit:
\begin{figure}[h]
\centering
\begin{minipage}{0.39\textwidth}\centering
    \begin{circuitikz}
        \draw (0,0) to[sV] (0,3) -- (3,3) to[R] (3,0) -- (0,0);
    \end{circuitikz}
\end{minipage}
\hfill
\begin{minipage}{0.59\textwidth}
\begin{lstlisting}
\usepackage{circutikz}
...
\begin{circuitikz}
    \draw (0,0) to[sV] (0,3) -- (3,3) to[R] (3,0) -- (0,0);
\end{circuitikz}
\end{lstlisting}    
\end{minipage}
\end{figure}

\verb|(0,0) to[component] (1,0)| adds \texttt{component} halfway between the two coordinates.
The component list is extensive, and therefore the suggestion is to use the documentation.
There you can also find the name of the anchor points, which will come in handy in the next couple of examples. 

\subsubsection{Adding labels}
To add values with units, the suggestion is to use the package \texttt{siunitx}.
In fact, this is recommended across any document.
Basic usage is \verb|\SI{value}{unit}|, and we have the option to either describe the units in text or simply using a symbol: \verb|\ohm| or \verb|ml|.

You have a couple of options with how to produce the label, \texttt{[element-name=label]}, or \texttt{[element-name, a=label]}.
It automatically chooses a good position, but it can be tweaked with \verb|a=| for one side, and \verb|l=| on the other.

\begin{figure}[h]
    \centering
    \begin{minipage}{0.39\textwidth}\centering
        \begin{circuitikz}
            \draw (0,0) to[sV=\SI{5}{\volt}] (0,3) -- (3,3) to[R, a=\SI{100}{\ohm}] (3,0) -- (0,0);
        \end{circuitikz}
    \end{minipage}
    \hfill
    \begin{minipage}{0.59\textwidth}
    \begin{lstlisting}
\usepackage{circutikz}
\usepackage{siunitx}
...
\begin{circuitikz}
    \draw (0,0) to[sV=\SI{5}{\volt}] (0,3) -- (3,3) to[R=\SI{100}{\ohm}] (3,0) -- (0,0);
\end{circuitikz}
    \end{lstlisting}    
    \end{minipage}
\end{figure}

\subsubsection{Adding an amplifier}
For this, we need to use the other way of adding components, by creating a \texttt{node} at a particular point.
We do so with \verb|\node[op amp]|, which results in the line being placed in the centre of the amplifier:
\begin{figure}[h]
\begin{minipage}{0.40\textwidth}\centering
    \begin{circuitikz}
        \draw[] (0,0) -- (3,0) node[op amp]{};
    \end{circuitikz}
\end{minipage}
\begin{minipage}{0.50\textwidth}
    \begin{lstlisting}
\begin{circuitikz}
    \draw (0,0) -- (3,0) node[op amp]{};
\end{circuitikz} 
    \end{lstlisting}
\end{minipage}
\end{figure}

To correct this to our needs, we set both an anchor point and indicate the non-inverting input to be on top:
\verb|node [op amp, noinv input up, anchor=+]{}|.

\begin{figure}[h]
\centering
    \begin{circuitikz}
        \draw (0,0) to[sV=\SI{5}{\volt}] (0,3) -- (3,3) to[R, a=\SI{100}{\ohm}] (3,0) -- (0,0);
        \draw (3,3) to[short,*-] ++ (2,0) node [op amp, noinv input up, anchor=+](opamp){};
        \draw (3,0) to[R,-*] (opamp.- |- 3,0) coordinate (intersection) to[R] (opamp.out |- intersection) -- (opamp.out) to[short,-o, l= \( v_{out} \) ] ++ (0.2,0);
        \draw (opamp.-) -- (intersection);
    \end{circuitikz}
\end{figure}

\begin{lstlisting}
\begin{circuitikz}
    \draw (0,0) to[sV=\SI{5}{V}] (0,3) -- (3,3) to[R, a=\SI{100}{\ohm}] (3,0) -- (0,0);
    \draw (3,3) to[short,*-] ++ (2,0) node [op amp, noinv input up, anchor=+](opamp){};
    \draw (3,0) to[R,-*] (opamp.- |- 3,0) coordinate (intersection) to[R] (opamp.out |- intersection) -- (opamp.out) to[short,-o, l= \( v_{out} \) ] ++ (0.2,0);
    \draw (opamp.-) -- (intersection);
\end{circuitikz}
\end{lstlisting}    

Remember: \verb|++ (x,y)| creates relative coordinates, so \verb|(1,1) -- ++ (2,0)| means a line from (1,1) to (3,1)

The usage of \texttt{|-} and \texttt{-|} may take some getting used to.
\texttt{(0,0) |- (1,1)} creates the lines \verb|(0,0) -- (0,1) -- (1,1)|.
Whereas \texttt{(0,0) |- (1,1)} creates the line \verb|(0,0) -- (1,0) -- (1,1)|.

On the other hand, \verb!(opamp.out |- intersection)! means ``the coordinates for the intersection of \texttt{opamp.out} and \texttt{intersection}''. 
Notice the parenthesis indicating coordinates!

\verb!(opamp.- |- 3,0) coordinate (intersection)!: At any point you can give coordinates an identifier.
The main motivation is that we don't know the absolute coordinates for this point, but need to refer to it later on.
Alternatively, if you use the same coordinate continuously, it may be worth naming it as well.

\verb|to[short]| and \verb|--| both produce lines, but the former has some really interesting uses.
Most importantly is the use of \verb|-*| and \verb|-o|.
You will notice that in the example we used the code below to create $v_{out}$.  
\begin{lstlisting}
(opamp.out) to[short,-o, l= \( v_{out} \) ] ++ (0.2,0)
\end{lstlisting}
Like with other components, this is creating the label above/below the midpoint of the component.
But maybe you want the label to be above the end point, which requires the use of a \texttt{node} at the end coordinates.
That is:
\begin{figure}[h]\centering
\begin{minipage}{0.40\textwidth}\centering
    \begin{circuitikz}
        \draw (0,0) to[diode,-o,l=above component] (2,0) ;
        \draw (0,1.5) to[Do,-o] (2,1.5) node[above]{end};
        \draw (0,-1.5) node[above]{start} to[leDo,-o] (2,-1.5);
    \end{circuitikz} 
\end{minipage}
\hfill
\begin{minipage}{0.59\textwidth}
    \begin{lstlisting}
\begin{circuitikz}
    \draw (0,0) to[diode,-o,l=above component] (2,0) ;
    \draw (0,1.5) to[Do,-o] (2,1.5) node[above]{end};
    \draw (0,-1.5) node[above]{start} to[leDo,-o] (2,-1.5);
\end{circuitikz} 
    \end{lstlisting}
\end{minipage}
\end{figure} 

\subsubsection{Other considerations}
There is a lot \texttt{circuitikz} is capable of which was not covered.
Do use the \href{https://mirror.ox.ac.uk/sites/ctan.org/graphics/pgf/contrib/circuitikz/doc/circuitikzmanual.pdf}{documentation} when you question how or if something can be done.

\begin{figure}[h]\centering
    \begin{minipage}{0.45\textwidth}
        \begin{circuitikz}[european voltages]
            \draw (0,0) node[ground]{}
            to[battery] ++ (0,1.5)
            to[sR, i^=$i_1$] ++ (3,0)
            to[C,v^>=$v_1$] ++ (0,-1.5) ;
        \end{circuitikz}
    \end{minipage}
    \begin{minipage}{0.45\textwidth}
        \begin{lstlisting}
\begin{circuitikz}[european voltages]
    \draw (0,0) node[ground]{}
    to[battery] ++ (0,1.5)
    to[sR, i^=$i_1$] ++ (3,0)
    to[C,v^>=$v_1$] ++ (0,-1.5) ;
\end{circuitikz}
        \end{lstlisting}
    \end{minipage}
\end{figure}

While creating a separate \verb|\draw;| for each drawing is not actually necessary, you are encouraged to do so for tidiness and legibility.
Perhaps separating into sections, continuous lines or however you see fit.

\begin{figure}[h]
    \begin{minipage}{0.45\textwidth}
        \begin{circuitikz}
            \node [op amp] (opamp) at (0,0) {};
            \draw
                (opamp.out) |- ++ (-1,1.5) to[R] ++ (-2,0) |- (opamp.-) to [R] ++ (-3,0) node[ground]{}
                (opamp.+) to [short,-o] ++ (-1,0) node[left]{$V_{in}$}
            ;
            \draw (opamp.out) to[short,-o] ++(0.5,0) node[right]{$V_{out}$};
        \end{circuitikz}
    \end{minipage}
    \hfill
    \begin{minipage}{0.45\textwidth}
\begin{lstlisting}
\begin{circuitikz}
\node [op amp] (opamp) at (0,0) {};
\draw
    (opamp.out) |- ++ (-1,1.5) to[R] ++ (-2,0) |- (opamp.-) to [R] ++ (-3,0) node[ground]{}
    (opamp.+) to [short,-o] ++ (-1,0) node[left]{$V_{in}$}
;
\draw (opamp.out) to[short,-o] ++(0.5,0) node[right]{$V_{out}$};
\end{circuitikz} 
\end{lstlisting}
    \end{minipage}
\end{figure}

Below you can see an example with only absolute coordinates.

\begin{figure}[h] \centering
    \begin{minipage}{0.44\textwidth}
    \begin{circuitikz}[scale=0.8]
        \draw (0,0.5) to[short, o-*] (0.65,0.5) |- (1.3,1) to[R=R] (3.3,1) to[R=R] (6.3,1) -- (6.3,0.5) to[short, *-o] (7,0.5);
        \draw (0.65,0.5) -- (0.65,0) to[C,a=C] (3.3,0) to[C,a=C] (6.3,0) -- (6.3,0.5);
        \draw (0,-3.5) to[short,o-o] (7,-3.5); %ground
        \draw (2.7,0) to[resistor, a=$\frac{R}{2}$, *-*] (2.7,-3.5);
        \draw (3.8,-3.5) node[circ]{} to[C, a=2C] (3.8,-0.2) -- ++ (0,0) arc (-90:90:0.2) -- (3.8,1) node[circ]{};
        \draw [|->] (0,-3.35) -- (0,0.35) node[midway,fill=white]{$V_{in}$};
        \draw [|->] (7,-3.35) -- (7,0.35) node[midway,fill=white]{$V_{out}$};
    \end{circuitikz}
\end{minipage}
\hfill
\begin{minipage}{0.55\textwidth}
\begin{lstlisting}
\begin{circuitikz}
    \draw (0,0.5) to[short, o-*] (0.65,0.5) |- (1.3,1) to[R=R] (3.3,1) to[R=R] (6.3,1) -- (6.3,0.5) to[short, *-o] (7,0.5);
    \draw (0.65,0.5) -- (0.65,0) to[C,a=C] (3.3,0) to[C,a=C] (6.3,0) -- (6.3,0.5);
    \draw (0,-3.5) to[short,o-o] (7,-3.5); %ground
    \draw (2.7,0) to[resistor, a=$\frac{R}{2}$, *-*] (2.7,-3.5);
    \draw (3.8,-3.5) node[circ]{} to[C, a=2C] (3.8,-0.2) -- ++ (0,0) arc (-90:90:0.2) -- (3.8,1) node[circ]{};
    \draw [|->] (0,-3.35) -- (0,0.35) node[midway,fill=white]{$V_{in}$};
    \draw [|->] (7,-3.35) -- (7,0.35) node[midway,fill=white]{$V_{out}$};
\end{circuitikz}
\end{lstlisting}
\end{minipage}
\end{figure}

It's quite easy to get lost with complicated circuits, so the suggestion is to think about it in terms of a continuous line drawing.
You will notice this circuit draws the horizontal lines across the the resistors, then across the capacitors, then the vertical lines. 