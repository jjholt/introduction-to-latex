\begin{enumerate}
    \item Using the \texttt{tikz} package, recreate the following drawing:
    \begin{figure}[h]\centering
        \newcommand{\x}{3}
    \begin{tikzpicture}
        \draw[-latex, thick] (0,0) -- (5,0) node[right]{$x$};
        \draw[-latex, thick] (0,0) -- (0,5) node[above]{$y$};
        \draw[] (0,0) node[below]{a} -- (\x/2,\x) node[above]{b} -- (\x+\x/2,\x)node[above]{c} -- (\x,0) node[below]{d} -- cycle;
    \end{tikzpicture}
    \end{figure}
    \begin{enumerate}
        \item In a \texttt{tikzpicture} environment, \texttt{draw} the axis lines with length of 5.
        
        \textbf{Hint:} Use the options \texttt{[-latex, thick]} to get this particular arrow style.
        \item Define \verb|\newcommand{\x}{3}|, then draw the parallelogram with vertices at coordinates \verb|(0,0)|, \verb|(\x/2,\x)|, \verb|(\x+\x/2,\x)| and \verb|(\x,0)|.
        \item Place a \texttt{node} at each vertex and give it a label.    
    \end{enumerate}
    \item Recreate the following control system
    \begin{figure}[h]\centering
        \begin{tikzpicture}
            %Blocks
            \node [block] (G) at (0,0) {G(s)};
            \node [block, right = 1.5 cm of G,] (controller) {Controller};
            \node [block, below left = 1 cm of controller] (feedback) {Feedback} ;
            \node [sum, left = of G] (sum) {};
            %Arrows
            \draw[->] (G) -- (controller);
            \draw[->] (controller.east) -- ++(1.0,0) node [right,above](){Output};
            \draw[->] (controller.south) |- (feedback.east) ;
            \draw[->] (feedback.west) -| (sum) node[below left](){+};
            \draw[->] (sum.west) ++ (-1,0) node[above](){Input}-- (sum.west)
                (sum.east) -- (G.west)
            ;
            \draw[->] (sum.north) ++ (0,0.8) node{Noise} -- (sum.north);
        \end{tikzpicture}
    \end{figure}

\textbf{Hints:} \begin{itemize}
    \item You may want to use the tikz library \texttt{positioning}. Include \verb|\usetikzlibrary{positioning}| in the preamble.
    \item Use either the template or \verb|\tikzstyle| to define the \texttt{block} and \texttt{sum} styles in the preamble:
    \begin{lstlisting}
\tikzstyle{block} = [draw, minimum width = 2cm, minimum height = 1.2cm, fill=blue!5]
\tikzstyle{sum} = [draw, fill=blue!5, circle, node distance=1cm]
    \end{lstlisting}
\end{itemize}
\end{enumerate}