\documentclass{article}
\usepackage{circuitikz}
\begin{document}
\begin{figure}[h]
    \centering
    % Bridge: Top upper corner
    \newcommand{\x}{4.5}
    \newcommand{\y}{6}
    % Bridge: Length
    \newcommand{\dx}{3}
    \newcommand{\dy}{3}
    \begin{circuitikz}
        \draw
        (0,0) to [sV, l=$A\cos(2\pi f_s t)$] (0, \y) %Voltage source 
        to (\x, \y) 
        % Left half bridge
        to [R, a=$Z$, *-*] (\x-\dx,\y-\dy) coordinate (bridge-left)% Top left resistor
        to [R=$Z$, -*] (\x,\y-2*\dy)  % Bottom left resistor
        ;
        \draw (\x,\y)
            % Right half bridge
            to [R=$L_0(1-x)$, -*] (\x+\dx, \y-\dy) coordinate (bridge-right) % Top right resistor
            to [R=$L_0(1+x)$, -*] (\x,\y-2*\dy)  % Bottom left resistor
            to (\x, 0) to (0,0) %Line to voltage source
        ;
        \draw
            (bridge-right) to [R, a=\(R_3\)] ++ (3,0) %R_3
            node [op amp, anchor=+](opamp) {} %Op Amp
            to [R=\(R_4\),*-] ++ (0,-2.5) %R_4
            node[ground]{} %Ground
        ;
        \draw
            (bridge-left) |- (\x+\dx, 1.2*\y)
            -- (opamp.- -| \x+\dx, 1.2*\y) %coordinates for intercept between opamp.- and (\x+\dx, 1.2*\y)
            to [R,a=\(R_1\),-*] (opamp.-) %R_1
            -- ++ (0,1.5) coordinate (z) %Call it z because we don't know where it is, but need the coordinates for next line 
            to [R=\(R_2\)] (opamp.out |- z) %R_2
            -- (opamp.out) %Line from R_2 to opamp.out
            to [short,*-o] ++ (0.5,0) node[right] {$V_0(t)$}
        ;
    \end{circuitikz}
\end{figure}
\end{document}